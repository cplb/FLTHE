\chapter{A course on proposal writing}\label{ap:Scriptoria}

As part of my continuing personal development, on 16 September 2014 I attended a one-day course on writing funding proposals. The course was organised by the University of Birmingham's People \& Organisational Development division and was delivered by Scriptoria.\footnote{Scriptoria offer consultancy on improving communications, and assistance designing, writing and editing publications as well as scientific-writing training courses \url{http://scriptoria.co.uk/}.} It was aiming for researchers from the College of Engineering \& Physical Sciences. My primary aim in attending was to improve my understanding of how to write grant proposals, a genre of technical writing I do not have experience of; however, the session serves a secondary purpose of giving an example of how to teach writing skills to physical scientists.

The course began with a discussion from senior academics who had served on grant-review panels explaining the refereeing process and what they look for in submissions. This was useful in establishing how proposals are read and clarifying the criteria for success. I found it reassuring to understand how a proposal would be assessed. The equivalent for my teaching of essay writing is to explain how I mark the essays; this is done as I explain the mark scheme (tutorial 14, \apref{plan}).

The remainder of the course was a mixture of lecturing, discussion and working through examples. This format is familiar from SGTs. The class size ($\sim10$--$20$) was larger than typical for my own SGTs, this was compensated for by splitting us to work in pairs or smaller groups \citep{Ayres2015}. Topics discussed included content, presentation, writing style, common English mistakes, how to write (the writing process), and editing. These would all be useful topics for essay-writing. However, this course was longer than the time I have available in tutorials, hence I cannot hope to cover the subject in as much detail (\secref{timetable}). I do not cover common English mistakes or how to write, and I only briefly discuss how to write when discussing how to plan and organise their work, as these are not priorities for my second-year tutorials.\footnote{I hope that students can figure out these areas for themselves; that it is beneficial to teach these areas to academics indicates that I may be overly optimistic. However, there are many approaches to writing \citep{Biggs1988}, so it may be counter-productive to dictate a pattern to my students.} The activities of reading through samples of work and then reviewing these, highlighting good areas and suggesting improvements, were useful; this could be considered as a peer learning exercise \citep[chapter 1]{Falchikov2001}. While beneficial, these activities were time consuming and there were difficulties in judging how long they would take; some I completed easily in the time, others I was only part way through. Consequently, they must be employed prudently in tutorials. Managing tutorial time and ensuring that activities cover an appropriate amount of material to a suitable depth is difficult (\secref{teach-n-learn}).

To supplement the class activities, and to act as a future reference, all participants were given a booklet. This contains advice, covering all the material discussed plus some additional material on using word-processing software, as well as the examples and exercises worked through. I found this helpful as it meant I could concentrate on understanding the discussions rather than making notes; the amount of content covered would have made it impossible to construct adequate notes, especially since exact wording is important. Having a resource to look back on is something that I will find beneficial when it comes to constructing and reviewing my own writing. This booklet helped to convince me that constructing a written guide, my blog posts (\secref{blog}), would be useful.

Overall, I believe the session did help me improve my proposal writing. This has the benefit of making me better qualified to teach scientific writing. More directly, this course gave me guidance for my own teaching.
