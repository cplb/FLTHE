\chapter{Introduction to tutoring physics}\label{ch:intro}

In this chapter we discuss the teaching and learning of physics with particular reference to my experiences. We learn from our own experiences, repeating things that have gone well, and avoiding those that led to failure \citep[chapter 2]{Skinner1954,Kolb1984}. When reflecting upon our own teaching, we must consider how our own perception is influenced by our own experience, and how our students may not share the same view;\footnote{The Johari window gives a simple way of visualising how one's own experience does not overlap with others', but one's outlook can be expanded through feedback \citep{Luft1961}. Sharing experiences (exposing one's own knowledge) with others can be considered an aspect of teaching \citep[cf.][chapter 7]{Ramsden1992}.} hence, we begin with a brief summary of my personal development to date in \secref{teach-n-learn}. We continue from here by introducing core components of studying physics (\secref{physics}), before expanding upon small-group teaching (\secref{small}), a primary component of both my learning and teaching experience. Building upon the ideas of learning in a small group, my experiences of learning through peer collaboration are discussed in \secref{peer}. The chapter concludes with a critique of how I have assimilated aspects of my own student experience into my approach to teaching.

\section{Teaching \& learning}\label{sec:teach-n-learn}

My own teaching style reflects how I learnt as I attempt to emulate the experiences that were most educational for me. Given that I was both sufficiently academically able and interested in my subject to continue to hold a university research position, it may be assumed that I do not reflect the average student. Therefore, while my personal experience is a sensible starting point, it is vital to remember that others may have different preferences \citep[chapter 5]{Ramsden1992} and those who need the most support will be those who find the subject more difficult. To try to understand other perspectives and broaden my outlook, it is necessary to seek feedback (see appendices \ref{ap:student} and \ref{ap:peer}) and research studies of educational practices (discussed throughout).

I read Natural Sciences at the University of Cambridge as an undergraduate, specialising in Experimental and Theoretical Physics, before continuing to complete a Ph.D.\ in Astronomy. During my postgraduate study I began supervising: tutoring small groups (2--3 students), with teaching focussed on weekly (formative) problem sets. These had been an integral part of my undergraduate study, hence I was familiar with their function and form. Following completion of my doctorate, I have moved to the University of Birmingham, where I now tutor: small-group (3--5 students) teaching similar to supervising, except that problems also play a summative role. At both Cambridge and Birmingham, I have taught second-year physics for these SGTs.

In the last academic year, I have continued my personal development by lecturing a short, optional course to a mix of fourth-year undergraduates and postgraduate students.\footnote{The course was approximately $52\%$ undergraduates, $34\%$ postgraduates and $14\%$ interested staff.} Lecturing will not be discussed in detail in this work, but there are many parallels between SGTs and lectures: explaining clearly; maintaining an environment suitable for learning; checking students' understanding and progress; planning material to deliver and doing so at good pace (within a fixed time); engaging students in discussions and interesting them in their studies, and seeking feedback from students (\citealt[chapter 3]{Brown1988}; cf.\ \citealt[chapter 6]{Ramsden1992}).\footnote{Tutorials can become too much like lectures \citep[chapter 9]{Ramsden1992}; in a study of SGTs in British universities, \citet[quoted in \citealt{Brown1988}, chapter 4]{Luker1987} found that between $7\%$ and $70\%$ of the time is spent by the tutor lecturing.} I found that my experience with SGTs made it more natural for me to engage with the class and have them work through problems, which proved popular. I was surprised at how difficult it was to judge how long it would take to cover a particular topic. I am used to going through problems and explaining concepts in tutorials, and keeping to an hour time-slot; however, extending this to a lecture where a single narrative, explained such that a large group can keep up, must be contained within a hour was challenging. This has made me more conscious of how I allocate time in tutorials, and in particular that I leave time for a summary at the end to recap what we have discussed.\footnote{The usefulness of an ending summary was highlighted through my experience of peer observations (both reflecting on feedback on my tutorials and while observing the lecturing of another) discussed in \apref{peer}.}

\section{Studying physics}\label{sec:physics}

Physics is a broad subject covering the fundamental behaviour of the Universe and its components. Familiar to most as it is introduced at school level, its teaching follows a spiral curriculum, periodically returning to previously-taught topics to cover them in ever-expanding detail \citep{Bruner1960}. The main jump from school to university physics is in the use of mathematics. Applying mathematical formalism to describe and solve physical problems is the greatest threshold concept \citep{Meyer2003} faced by those wishing to become conversant with physics \citep{Wigner1960,Roth1994}.

My experience of university-level physics teaching has been traditional \citep[cf.][]{Iannone2015}. Concepts and theories are introduced in lectures. These ideas are reinforced through problem sets that also encourage mathematical fluency and independent learning \citep{Pike2015}. At least some of these problems are assessed to provide students with feedforward information \citep{Bloxham2015}. SGTs or examples classes give students the opportunity to discuss the problems and their solutions, as well as other elements of the syllabus.\footnote{This may fulfil the role of inverted classrooms, where students prepare outside of class and then work through problems under the guidance of a tutor in class, employed elsewhere \citep{Lage2000}.} Practical work is done in labs; experiments may complement topics covered in lectures;% for example measuring refraction during a course on electromagnetism;
be of equal importance as lectures for teaching ideas, %for example in a course on electronics,
or motivate material covered in lectures %, for example error analysis
\citep{Hanif2009}. Assessment is dominated by summative closed-book examinations at the end of the courses.\footnote{The situation may be comparable to that in mathematics where traditional closed-book examinations are favoured \citep{Iannone2014}, as both subjects require similar (mathematical) problem solving and the correctness of solutions is not subjective.} I have found completing problem sets to best encourage deep learning \citep{Marton1976,Marton1976a}: good understanding is required to fully solve a question and, since problems are done outside of class, there is sufficient time to review and assimilate new ideas.\footnote{Even if a student takes an achieving strategy \citep[chapter 2]{Biggs1987} and only invests time in understanding the topics directly probed by the given questions, a dense covering of the syllabus by problems sets should ensure a comprehensive understanding of the majority of the material.}

Problem sets are of paramount importance, both for teaching subject material and key skills such as problem solving and mathematical proficiency. Extracting the most from them relies on the student being motivated to put in the necessary effort: those who work harder shall learn more \citep{Gibbs2015}. This is difficult to externally inspire \citep[cf.][]{Ryan2000}, especially since a student taking a surface-learning approach is more likely to find problems tedious and unprofitable, potentially pushing them further away from engaging with the material \citep[chapter 4]{Ramsden1992}. Consequently, we shall instead focus on how problem sets are supported, which is primarily through SGTs.

\section{Small-group teaching}\label{sec:small}

Supervisions at Cambridge involve either an academic or a Ph.D.\ student meeting regularly with a group of students. The small group size allows for sessions to be tailored to the individual needs of the students and for each student to be involved in discussions.\footnote{Groups of only one student are usually avoided as this could be intimidating for the student. Furthermore, having an additional student provides some respite to think while the other is being questioned. The lone student would also miss out on the opportunity to hear the opinions of a peer.} Supervisions are generally structured around the problems, with the students given opportunity to ask about other areas of the course they were unsure of (or interested in); any spare time is used to discuss additional topics (perhaps the research of the supervisor). The supervision system is designed to encourage students to stay on top of their work and to provide them with the necessary support when this is difficult; the overall effect is to promote deep learning \citep[case study 14.1]{Gibbs2015}.

Tutorials in Birmingham are of a similar nature to supervisions. They differ in group size, which does not afford as intensive attention to individual students, and also in the amount of contact time. In second year, Cambridge's Natural Sciences undergraduates would have three hours per week (one per course), while Birmingham's Physics undergraduates have a single hour per week for all their courses.\footnote{The difference in teaching time is slightly ameliorated by the marginally longer teaching year in Birmingham: 22 weeks compared to 19.} While the tutorial system might not match supervisions in Cambridge, they still provide a great opportunity for students to engage with and ask questions of a member of research staff, an opportunity that is not present as part of all physics courses.\footnote{\citet{Sharma2007} notes how SGTs are of particular importance to physics students, with the interactivity and development of problem-solving skills praised, but that they are not present at all institutions through all years of study.}

A further difference is that tutorials in Birmingham include the (formal) teaching of key communication skills. These are part of the curriculum at Cambridge, but are not explicitly taught as part of supervisions. %(although students are still at liberty to ask about these topics).
I believe that teaching of these soft skills are to be encouraged as they are of universal utility and a necessity for employment; one of the objectives of higher education should be to prepare students for life and world outside of study \citep{Fallows2000,Harvey2000}. Communication skills enhance the employability of the students; they are not traditionally emphasised within a physics course \citep[cf.][]{Sharma2007}, but are considered a key area of competency for a physicist \citep{Gonsalves2014a}. However, including teaching of communication skills in tutorials also absorbs time that could have been spent on the main curriculum, further limiting the time for discussing problem sets or lecture materials. Striking a balance between different aspects of the curriculum is difficult.

For any SGT it is necessary for students to prepare. I did this by collaborating with my peers, which has had a significant impact on my outlook on physics education.

\section{Peer learning}\label{sec:peer}

As an undergraduate, I would often work on problems with a group of fellow physicists. Doing so provided support for difficult questions, an opportunity to discuss concepts, and multiple points-of-view. It made the experience much more enjoyable, and motivated continued effort \citep[e.g.,][chapter 1]{Roth1994,Springer1999,Falchikov2001}: working on questions was a social activity, and there was friendly competition to finding (the best) solutions. Cooperative learning has been found to improve student attainment (\citealt{Qin1995}; \citealt[chapter 2]{Falchikov2001}; \citealt{Cabrera2002}), and this particular example can be considered as an informal version of peer instruction \citep{Roth1994}, which has been demonstrated to be effective in encouraging learning in physics \citep{Springer1999,Crouch2001,Pilzer2001,Miller2006} and deep learning in general \citep{Marton1976,Wilson2005}.

Not only did working as part of a group improve our understanding of physics, but it also enhanced our communication skills: we became accustomed to explaining physics and our solutions. This is a vital skill for teaching physics, and I believe that the experience I accumulated during my undergraduate studies have greatly benefited my teaching (see sections \ref{sec:A-topic} and \ref{sec:B-topic}).

\section{Translation from learning to teaching}

Several elements from my own studies have been incorporated into my current teaching style. From Cambridge I have inherited high expectations for my students' work. This helps stretch them to achieve their best, and engage them by challenging them academically \citep{Bamber2015}. I explain my expectations with students during our first meeting (see \apref{plan}) to make sure these are explicit \citep{Butcher2015}, and to try to ensure that the students appreciate the standards required of them \citep[chapter 8]{Ramsden1992}. Many students have expressed approval of having high targets set, either because they are eager to be challenged or because they like to have clear goals to motivate themselves; of course, this does not translate to students always being adequately prepared for tutorials; usually there is a decline as the term progresses and students become fatigued or over-burdened with work (see \secref{timetable}).

Cambridge also gave me much experience of discussing physics, both in supervisions and working outside. In my SGTs, engaging students in conversation is important: they need to actively discuss concepts, and ask questions freely. While I cannot engineer an environment that matches my own experience of having a circle of friends to study with, I can try to make tutorials a venue for friendly conversation. To aid this, I adopt an informal style, mirroring how I would interact with my peers. I employ humour, incorporating jokes and anecdotes (about my own studies and famous physicists). Use of humour can improve retention and ease anxiety \citep[e.g.,][and references therein]{Korobkin1988,Lesser2008}. The latter improves group rapport, which makes students more comfortable contributing to discussions and, in particular, asking questions on areas of weakness. Even though I cannot arrange for my students to benefit from the same collaborative learning that I enjoyed, I try to make sure that they can obtain similar benefits from my teaching; in doing so, I hope to establish a good teacher--student relationship, which can have positive learning outcomes \citep{Cornelius-White2007}.

Having now established some of the core components of undergraduate physics education and discussed how my own education has influenced my teaching, we will move on to look at specific aspects of my SGTs, and how these are designed and evaluated.
