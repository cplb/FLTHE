\chapter{Student evaluation}\label{ap:student}

Teaching should not be considered independently from learning; when evaluating the effectiveness of a teaching practice it is necessary to listen to feedback from students and peers \citep[chapter 7]{Ramsden1992}. In this appendix, I discuss feedback from my students. In \apref{peer} I include feedback on my teaching from peers. Students are well-placed to evaluate their own learning, and frequently values the same things as teachers \citep[chapter 6]{Ramsden1992}.

In \secref{2014-15students} I describe my students and in \secref{questions} I explain how I collected feedback from them. This feedback is analysed in \secref{results}, and I summarise my conclusions in \secref{student-conc}.

\section{2014/2015 students}\label{sec:2014-15students}

This academic year I taught a total of nine students; I had one tutorial group of four and another of five. The fifth member of the second group joined at the start of the Spring term. Since their experience is anomalous and they did not complete the practice essay, their feedback is not included in the analysis presented here. Amongst the remaining eight students, two are female (one in each group), one is an international student (with English as a second language), and none have disclosed special learning needs. The students represent a range of abilities as assessed from their first-year marks, covering the span from 2.ii to 1st.

\section{Collecting feedback}\label{sec:questions}

\subsection{Overview}

I asked for feedback from my students at the start of the Spring term to coincide with the PRTs (see \apref{plan}). I tried to link the ideas of reviewing my progress with reviewing theirs to make the topic more approachable. I asked them to complete an anonymous, electronic questionnaire. This is explained in \secref{form1}. Most did this ahead of the PRT, so I was able to ask them some additional follow-up questions, as well as asking them if there was anything further they would like to discuss.\footnote{\citet[chapter 11]{Ramsden1992}, emphasised that one must not use questionnaires blindly to assess quality of teaching.}

I asked the students to complete a further, shorter questionnaire at the end of the Spring term, once they had received feedback from their essays. This is explained in \secref{form2}. I have not had opportunity to follow this up; however, I will ask for any further comments at the end of the year.

\subsection{First Questionnaire}\label{sec:form1}

\subsection{Second Questionnaire}\label{sec:form2}

\section{Results and discussion}\label{sec:results}


Gender \citep{Gonsalves2014,Gonsalves2014a}

Overwork \secref{B-presentation}.

\section{Conclusion}\label{sec:student-conc}
