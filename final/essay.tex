\chapter{Teaching essay writing}\label{ch:essay}

In this chapter we examine my teaching of essay writing. SGTs have been my primary experience of teaching (\secref{teach-n-learn}) and are an important component of an undergraduate physics course (\secref{small}) making them the natural subject for me to study my teaching. There are many aspects of tutorials, too many to examine in detail. The teaching of communication skills is one aspect that differs between my previous teaching in Cambridge and my current teaching in Birmingham. I have therefore selected essay writing, one component of this, as a subject. Conclusions drawn from how I teach essay writing should transfer to the teaching of other parts of the curriculum in tutorials.

We begin with a description of essay writing within the second-year course in \secref{problem}. Then we look at how I teach essay writing  (\secref{essay-teach}). I discuss my approach across both years of giving tutorials. I was dissatisfied with the effectiveness of my teaching in the first year, in particular I was concerned that some topics such as referencing, although covered in detail, but remained problematic. I identified, with help from peer discussion (\secref{ALS}), several areas where teaching could be enhanced, and decided to introduce two new elements to my teaching: setting a formative, practice essay (\secref{formative}) and providing written advice that can be used as reference (\secref{blog}). The outcomes of my teaching are evaluated in \chapref{conc}.

\section{The second-year essay}\label{sec:problem}

Second-year Physics students at the University of Birmingham must write an essay as part of their (compulsory) Physics \& Communication Skills 2 module, constituting $10\%$ of the marks for the module, which in turn makes up $10$ out of $120$ credits for the year. This means the essay is worth just under $1\%$ of the students' second-year grade. Students will have previously written an essay in first year as part of their skills training; this makes up $15\%$ of a $10$ credit module and shares an identical mark scheme to the second-year essay. The essay assessment emphasises writing proficiency over scientific content; the mark scheme is described in \tabref{mark-scheme}.\footnote{Taken from the Year 2 Physics and Communication Skills handbook \url{https://intranet.birmingham.ac.uk/eps/documents/staff/physics/students/handbook/handbook-y2-comms.pdf}.}
\begin{table}\scriptsize
\centering
\begin{tabular}{c c p{3in}}
\toprule
\multicolumn{1}{c}{Mark subsection} & \multicolumn{1}{c}{Category} & \multicolumn{1}{c}{Description} \\
\midrule 
\multirow{4}{*}{Content} & \multirow{1}{*}{Topic}  & Conceptually challenging and well treated $\leftrightarrow$ Inappropriate and poorly treated \\
			 & Reading	  & Wide range of source material $\leftrightarrow$ Little evidence of research \\
			 & Understanding  & Clear grasp of material $\leftrightarrow$ Uncritical rehash \\
\midrule
\multirow{9}{*}{Style}	 & Structure 	  & Introduction, development of topic and conclusion. \\
			 & English  	  & Grammar, paragraphs, sentences, spelling, etc. \\
			 & Basic writing style  & Mature, rigorous and scientific. \\
			 & \multirow{1}{*}{Clarity}  & Logical development, appropriate level, definition of terms, etc. \\
			 & \multirow{1}{*}{Presentation}  & Standard of word processing/typographical errors/layout of title, fonts and diagrams. \\
			 & \multirow{1}{*}{Referencing}  & Appropriate referencing of paper and electronic sources, including figures. \\
\midrule
\multirow{1}{*}{Discretionary}	 & Overall impression & --- \\
 \bottomrule
\end{tabular}
\caption{Mark scheme for the (first- and) second-year essay. Each category is worth $10$ marks, giving a total of $100$. The emphasis is on good scientific writing ($60\%$ of the marks) rather than on scientific content ($30\%$). Being outside of the word limits results in a loss of $10\%$ from the mark, and late submission results in a penalty of $10\%$ per day. Students who score below $60\%$ have to resubmit a revised version within two weeks; failure to do so results in a score of zero. Resubmitted essays are capped at a maximum mark of $60$.}\label{tab:mark-scheme}
\end{table}

The essay can be on any physics-related topic (except those used for the first-year essay and their oral presentation). The students must research this themselves. The word limit is $2000 \pm 200$. 

In addition to the essay, students have other pieces of assessed, extended scientific writing associated with their projects (another $10$ credit module). They must write a preliminary report of $2$--$4$ sides of A4 by the third week of the Spring Term (see \apref{plan}). This is worth $10\%$ of the module and its main purpose is to ensure that students have a plan for their projects. Accordingly, written style is not too important provided that students can produce a concise and technical report. Following completion of their essay, they have to submit a write-up of their project, with a maximum length of $20$ sides of A4 and is worth $50\%$ of the module. Marks are awarded both for the quality of the project (judged from the report) and quality of the report itself. This project report is both longer and worth more marks than the essay; using the essay as a means of improving scientific writing is beneficial for the student. The essay must be submitted by the end of week 6 of the Spring Term (see \apref{plan}) and the project report by the end of week 11; this gives a short time to receive and digest feedback from the essay, making it advantageous for students to receive guidance and to think about their writing style earlier in the year.


\section{Teaching plan}\label{sec:essay-teach}

In this section we discuss my teaching of essay writing, beginning with a description of some of the key learning outcomes I have identified (\secref{essay-aims}). Then we discuss how the essay fits into the schedule for tutorials (\secref{timetable}) and some of my specific approaches to teaching (\secref{essay-methods}). I discuss my teaching last year (\secref{teach2013-14}) and this year (\secref{teach2014-15}), and highlight two new aspects (sections \ref{sec:formative} and \ref{sec:blog}).

\subsection{Objectives}\label{sec:essay-aims}

Scientific writing (as an essay, report or paper) has many aspects, as indicated in \tabref{mark-scheme}. Most fundamental is the ability to communicate clearly using English properly. Students should have a basic proficiency with written English before they start university; in practice, English levels may not be as high as is desirable, particularly for international students who are not native speakers and understandably find it more difficult.\footnote{Support for improving written English is available from the Academic Skills Centre run by the University of Birmingham Library Services \url{https://intranet.birmingham.ac.uk/as/libraryservices/library/skills/asc/index.aspx}, and English-language support for international students is available from the English for International Students Unit \url{https://intranet.birmingham.ac.uk/as/eisu/insessional/index.aspx}} In tutorials I focus on aspects of academic, scientific writing that are different from less formal writing and so are less familiar.

I will split the differences between scientific writing and more general written English into three classes:
\begin{enumerate}
\item Specialisations---Modifications of familiar concepts to match the expectations of formal scientific writing, e.g.\ the structure of an introduction, main body and conclusion.
\item Conventions---Rules for particular aspects of the writing that may be simple to follow, but could be overlooked if not explained, e.g.\ the proper typesetting of mathematics.
\item Threshold concepts \citep{Meyer2003}---The most challenging ideas to grasp as they require a transformation of how the work is viewed, e.g.\ the need for referencing.
\end{enumerate}
The first is easiest to teach and the third is most difficult, although all three are important and require attention. Specialisations include adopting an appropriate structure, using a formal tone and scientific language (aspects of the Structure, Basic writing style and Clarity mark-scheme categories). Many of these concepts may be familiar and students may just need to extend their current practices. Conventions are often arbitrary and may not be immediately apparent to the uninitiated, such as how to typeset terms in equations, how to label graph axes, or how to format entries in a bibliography. These primarily fall under the heading of Presentation in the mark scheme; following them makes the writing more recognisable as a piece of scientific text. Once students are aware of the conventions, following the rules is not challenging, but it may be difficult to convince the students that this is worth the effort. Threshold concepts are the ideas that have the largest impact on the development of the students and are discussed further in \apref{threshold}. For my students, I have found three threshold concepts:
\begin{enumerate}
\item Not restricting themselves to prose---Using equations, diagrams or graphs, tables or even bullet points to communicate information (depending upon the audience).
\item Referencing \citep{Warner2011}---Including in-text citations to appropriate sources in the relevant places to support and supplement their arguments.
\item Being quantitative---Using numbers (either empirical or theoretical) to make specific quantitative statements, either to replace potentially subjective qualitative statements or to provide unambiguous evidence for a statement, e.g. saying ``the surface of the Sun is $6000~\mathrm{K}$'' rather than ``the surface of the Sun is hot''.\footnote{Both a cup of tea and the centre of the Sun could equally be claimed to be hot but both have temperatures that are orders of magnitude different from the surface of the Sun. The lack of quantitative statements could be viewed as the surface manifestation of a deeper problem of failing to understand that a reader who wants to engage with the material, to evaluate the validity of the arguments or to perform their own calculations, would require this information.}
\end{enumerate}
Identifying means to make learning these concepts easier could potentially have a significant impact on the quality of students' written work.

\subsection{Scheduling}\label{sec:timetable}

Given infinite time, all of the previous subsection's concepts could be perfectly taught. However, tutorials are finite and must cover more than writing skills (\secref{small}). In \apref{plan} there is an outline for tutorials through the year. Students must not be over-burdened as this is detrimental to their learning \citep[chapter 8]{Ramsden1992}, and a perceived heavy workload is correlated with adoption of a surface-learning approach \citep[and references therein]{Kember1998}.\footnote{\citet{Kember1998} find the workload is perceived as higher by students learning in a second language, which makes it especially important not to schedule too much for international students.}

In the Autumn term, students must prepare oral presentations (also worth $10\%$ of the Physics \& Communication Skills 2 module), which must be delivered by the end of the term. This leaves little time for written communication skills. However, it is possible to link the two types of communication skills together: there is significant overlap between the two (thinking of audience, planning structure, presenting technical content, conducting research, etc.). Thus, preparing and delivering the presentation can be used as instruction for the essay provided that the connections are made apparent. This integration has the additional benefit of ensuring that the presentation is not viewed as an isolated task, encouraging students to use feedback from this experience \citep{Housell2003}.

Following completion of the presentations, students have the Christmas vacation and the first half of the Spring term for their essays. The vacation provides a good opportunity to pick a topic and begin research, and writing skills can be discussed when they return. The Spring tutorials will be where most of the teaching is done.

While the presentation precludes spending time on writing skills for much of the Autumn term, there remains some freedom at the start of the year. In the first few weeks, students' workloads are lower, since problem sets have yet to be issued; consequently there is less to be discussed during tutorials. Students have to provide an updated CV by the third tutorial of term, but this is a small task that can be accomplished quickly.\footnote{The students write a CV in first year, so it only requires a few modifications to bring it up to date. Discussing their CV in the first one-on-one meeting (see \apref{plan}) is a useful way to get to know the students and it leads naturally onto their academic background, how they did last year, and where they hope to be in the future.} This time at the start of the year could be filled with material relating to the essay.

\subsection{Methods \& activities}\label{sec:essay-methods}

\subsubsection{Tutorials \& feedback}

Teaching of essay writing has followed a similar design to any other curriculum item. I discuss items with the students to keep pace with their work, and work through examples on the board. They are free to ask questions and suggest topics that require particular attention. I suggest other sources of information or useful activities for outside of tutorials, and draw connections to other areas of the syllabus. Following completion of the work, I provide feedback and we discuss troublesome areas.

My own experience of being taught writing skills follows a similar pattern, as detailed in \apref{Scriptoria}.

During discussions, specialisations arise naturally. Students know that these are things to ask about and potentially develop further. Following initial questions, I cover any elements they had not thought of. This leads naturally into me bringing up conventions as I can highlight them as something they are probably unaware of. Working through some model answers is intended to help them to digest this knowledge \citep[chapter 10]{Ramsden1992}. Larger topics (threshold concepts) require more attention and detailed explanations planned ahead of the tutorial.

Following submission, I try to provide detailed feedback. For each student, I ensure there is one aspect that was given specific praise \citep{Henderlong2002}. Together with their mark for each category (\tabref{mark-scheme}) I give comments highlighting good areas and those for improvement. On the essays themselves I give detailed comments, highlighting good and bad practices, using two different colours: one colour for smaller matters such as spelling or typographical errors, which are not normally considered important feedback \citep[chapter 4]{Irons2008} but are emphasised in this mark scheme, and another colour for the more complex issues, such as comments on physics, logical development or essay structure. The feedback is discussed at length in tutorial, where specific examples are examined and students can ask questions \citep[chapter 2]{Irons2008}.

\subsubsection{2013/2014}\label{sec:teach2013-14}

The first year I gave tutorials, my essay preparation consisted of three main components (\tabref{2013-14}): asking the students to begin researching and planning their essay over Christmas, to be discussed in the first week of term; a main discussion of scientific writing with special emphasis on referencing, which they highlighted as an area of concern (see \apref{threshold}), in the third week of term, and a final session to recap points I had made earlier (and to expand on conventions) and to ensure that students are familiar with the mark scheme (\citealt[chapter 8]{Ramsden1992}; \citealt{Bell2012}) in the fifth week of term. These three sessions roughly align with the three stages of essay construction proposed by \citet{Biggs1988}: intentional (establishing aspirations prior to writing), parawriting (planning and updating knowledge), and writing (composing, reviewing and revising).

To try to make the assessment less stressful \citep[chapter 10]{Ramsden1992}, I offered office hours for any last-minute problems or questions to be discussed one-on-one. Additionally, I suggested students handed in their essay to me a day early, such that there was time to correct any common, trivial mistakes (such as forgetting the word count or missing a page) and subsequently have time to grapple with any technical problems of submitting electronically.

I was slow in delivering feedback, which is suboptimal \citep[chapter 4, and references therein]{Irons2008}, providing it in week 10. This was not only disadvantageous for the students to learn from the assessment, but also meant that it would be difficult to incorporate feedback into the writing of their project report. Furthermore, I had not anticipated that any students would need to resubmit, yet this was the case, and had to be done during the Easter vacation when it was not easy for them to talk to me.

\subsubsection{2014/2015}\label{sec:teach2014-15}

This academic year, I made several changes to my teaching (\tabref{2014-15}). First, I introduced a formative essay. The discussion following this, pre-empted much of the material presented in the twelfth 2013/2014 tutorial. In the equivalent tutorial this year, we recapped these points briefly before spending more time on the threshold concepts (primarily referencing). The discussion spilled over into the next tutorial, where the conventions of presenting mathematics were explained, which also gave a chance to link with the concept of including equations in the text. Although the tutorial discussions have been expanded and spread throughout the year, their character has not changed.

Feedback was provided more promptly \citep{Gibbs2015}. This should give it more impact and make it more likely to be assimilated as it becomes feedforward for the project report \citep{Housell2003,Bloxham2015}.

\subsubsection{Formative essay}\label{sec:formative}

In 2013/2014, following feedback discussions, several students expressed regret that they had not appreciated concepts prior to assessment. This led naturally to the idea of setting a formative essay and providing feedback on this earlier in the year. Formative assessments have many advantages: students are more likely to be open with their concerns; experiment and take risks; be motivated by their own learning, and develop their self-assessment \citep[chapter 1, and references therein]{Irons2008}. Additionally, they can help establish expected standards \citep[chapter 10]{Ramsden1992}. The practice essay also gives me opportunity to diagnose troublesome issues.

The main barrier to introducing a formative essay was the additional work \citep[chapter 4]{Irons2008}. Given time-table constraints (\secref{timetable}), the only gap was at the start of the year. I asked the students to produce a short essay on any topic such that they did not have to worry about research, and could concentrate on writing.\footnote{I specified that they must still include references.} The disadvantage of an early formative assignment was the disconnect from the summative one. The solution was to link writing skills with presentation skills: the formative essay feeds into the presentation which then feeds into the summative essay (which feeds into the project report). Teaching of communication skills is distributed throughout the entire year, with no gaps where they become irrelevant.

I provided (prompt) feedback on the formative essay as I would for the summative essay except I did not give marks. This was then discussed in tutorial where I also highlighted my blog as a useful reference for the material we discussed.

As an additional source of formative feedback, I also offered to provide feedback on essays from first year, and a couple of students took advantage of this.

\subsubsection{Written resources}\label{sec:blog}

To complement the feedback from the formative essay, I produced a blog post containing advice and examples \citep[chapter 4]{Irons2008}.\footnote{This is also time-efficient for me, as the same advice can be recycled for multiple years \citep[chapter 4]{Irons2008}.} This can be read at \url{http://cplberry.com/right-good/}. It can be used as a reference following the tutorial and I hoped students would refer to it when writing (see \apref{Scriptoria}). The possibility to review and ask questions (via comments) within their own time can help students, particularly non-native speakers, absorb the material \citep{Rainsbury2003}.

The blog post was well-received (sections \ref{sec:views-blog} and \ref{sec:me}), so I composed a second on the conventions of typesetting mathematics to reinforce the material from the thirteenth tutorial (\apref{plan}). This can be read at \url{http://cplberry.com/equaquette/}. Providing a single, easy-to-find reference for conventions, specifically designed for these students, should assist their learning \citep[cf.][chapter 9]{Ramsden1992}; the effectiveness of the blog is reviewed in the next chapter.

