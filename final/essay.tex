\chapter{Teaching essay writing}\label{ch:essay}

In this chapter we shall examine in detail an my teaching of essay writing. SGTs have been my primary experience of teaching (\secref{teach-n-learn}) and are an important component of an undergraduate physics course (\secref{small}) making them the natural subject for me to study my teaching. There are many aspects of tutorials, too many to examine in detail. The teaching of communication skills is one aspect that differs between my previous teaching in Cambridge and my current teaching in Birmingham. I have therefore selected essay writing, one component of this, as a subject. Conclusions drawn from how I teach essay writing should transfer to the teaching of other parts of the the curriculum in tutorials.

We begin with a description of essay writing within the second-year course (\secref{problem}), then we look at how the essay fits into the schedule for tutorials and how I have structure my teaching (\secref{timetable}). I have identified two areas where I could potentially improve my teaching of essay writing: setting a formative, practice essay and providing written advice that can be used as reference. These are described in sections \ref{sec:formative} and \ref{sec:blog} respectively. The outcomes of my teaching are evaluated in \chapref{conc}.

\section{The second-year essay}{\label{sec:problem}

Second-year Physics students at the University of Birmingham have to write an essay as part of their (compulsory) Physics \& Communication Skills 2 module. The essay constitutes $10\%$ of the marks for the module, which in turn makes up $10$ out of $120$ credits for the year. This means the essay is worth just under $1\%$ of the students' second-year grade. Students will have previously written an essay in their first year \ldots

The essay assessment emphasises writing proficiency over scientific content. The word limit is $2000 \pm 200$.

In addition to the essay, students have one other piece of assessed, extended scientific writing. Following completion of their essay, they have to submit a write-up of their project. This report has a maximum length of $20$ sides of A4 and is worth $50\%$ of the Physics Project module, which is also $10$ credits. The project report is both longer and worth more marks than the essay, hence using the essay as a means of improving scientific writing is beneficial for the student. The essay must be submitted by the end of week 6 of the Spring Term (see \apref{plan}) and the project report by the end of week 11; this gives a short time to receive and digest feedback from the essay, making it advantageous for students to receive guidance and to think about their writing style earlier in the year.

\section{Scheduling}\label{sec:timetable}

Not overburden

\section{Formative essay}\label{sec:formative}

\section{Written resources}\label{sec:blog}
