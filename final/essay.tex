\chapter{Teaching essay writing}\label{ch:essay}

In this chapter we shall examine in detail an my teaching of essay writing. SGTs have been my primary experience of teaching (\secref{teach-n-learn}) and are an important component of an undergraduate physics course (\secref{small}) making them the natural subject for me to study my teaching. There are many aspects of tutorials, too many to examine in detail. The teaching of communication skills is one aspect that differs between my previous teaching in Cambridge and my current teaching in Birmingham. I have therefore selected essay writing, one component of this, as a subject. Conclusions drawn from how I teach essay writing should transfer to the teaching of other parts of the the curriculum in tutorials.

We begin with a description of essay writing within the second-year course (\secref{problem}), then we look at how the essay fits into the schedule for tutorials and how I have structure my teaching (\secref{timetable}). I have identified two areas where I could potentially improve my teaching of essay writing: setting a formative, practice essay and providing written advice that can be used as reference. These are described in sections \ref{sec:formative} and \ref{sec:blog} respectively. The outcomes of my teaching are evaluated in \chapref{conc}.

\section{The second-year essay}{\label{sec:problem}

Second-year Physics students at the University of Birmingham have to write an essay as part of their (compulsory) Physics \& Communication Skills 2 module. The essay constitutes $10\%$ of the marks for the module, which in turn makes up $10$ out of $120$ credits for the year. This means the essay is worth just under $1\%$ of the students' second-year grade. Students will have previously written an essay in their first year as part of their skills training; this makes up $15\%$ of a $10$ credit module and shares an identical mark scheme to the second-year essay. The essay assessment emphasises writing proficiency over scientific content, the mark scheme is described in \tabref{mark-scheme}.\footnote{Taken from the Year 2 Physics and Communication Skills handbook \url{https://intranet.birmingham.ac.uk/eps/documents/staff/physics/students/handbook/handbook-y2-comms.pdf}.}

\begin{table}\scriptsize
\centering
\begin{tabular}{c c p{3in}}
\toprule
\multicolumn{1}{c}{Mark subsection} & \multicolumn{1}{c}{Category} & \multicolumn{1}{c}{Description} \\
\midrule 
\multirow{4}{*}{Content} & \multirow{2}{*}{Topic}  & Conceptually challenging and well treated $\leftrightarrow$ Inappropriate and poorly treated \\
			 & Reading	  & Wide range of source material $\leftrightarrow$ Little evidence of research \\
			 & Understanding  & Clear grasp of material $\leftrightarrow$ Uncritical rehash \\
\midrule
\multirow{9}{*}{Style}	 & Structure 	  & Introduction, development of topic and conclusion. \\
			 & English  	  & Grammar, paragraphs, sentences, spelling, etc. \\
			 & Basic writing style  & Mature, rigorous and scientific. \\
			 & \multirow{2}{*}{Clarity}  & Logical development, appropriate level, definition of terms, etc. \\
			 & \multirow{2}{*}{Presentation}  & Standard of word processing/typographical errors/layout of title, fonts and diagrams. \\
			 & \multirow{2}{*}{Referencing}  & Appropriate referencing of paper and electronic sources, including figures. \\
\midrule
\multirow{1}{*}{Discretionary}	 & Overall impression & --- \\
 \bottomrule
\end{tabular}
\caption{Mark scheme for the (first- and) second-year essay. Each category is worth $10$ marks, giving a total of $100$. The emphasis is on good scientific writing ($60\%$ of the marks) rather than on scientific content ($30\%$). Being outside of the word limits results in a loss of $10\%$ from the mark, and late submission results in a penalty of $10\%$ per day. Students who score below $60\%$ have to resubmit a revised version within two weeks; failure to do so results in a score of zero. Resubmitted essays are capped at a maximum mark of $60$.}\label{tab:mark-scheme}
\end{table}

The essay can be on any physics-related topic (except those used for the first-year essay and their oral presentation). The students must research this themselves. The word limit is $2000 \pm 200$. 

In addition to the essay, students have others piece of assessed, extended scientific writing associated with their projects (another $10$ credit module). They must write a preliminary report of $2$--$4$ sides of A4 by the third week of the Spring Term (see \apref{plan}). This is worth worth $10/%$ of the module and its main purpose is to ensure that students have a plan for their projects. Accordingly, written style is not too important provided that students can produce a concise and technical report. Following completion of their essay, they have to submit a write-up of their project. This report has a maximum length of $20$ sides of A4 and is worth $50\%$ of the module. Marks are awarded both for the quality of the project (judged from the report) and quality of the report itself. This project report is both longer and worth more marks than the essay, hence using the essay as a means of improving scientific writing is beneficial for the student. The essay must be submitted by the end of week 6 of the Spring Term (see \apref{plan}) and the project report by the end of week 11; this gives a short time to receive and digest feedback from the essay, making it advantageous for students to receive guidance and to think about their writing style earlier in the year.


\section{Teaching plan}

In this section

\subsection{Objectives}

Threshold concepts

\subsection{Workload}\label{sec:timetable}

Essay writing is concentrated in the spring term. The essay is due by the end of the sixth week (see \apref{plan}). Talk.

Not overburden

\subsection{Methods and activities}

Praise specific areas \citep{Henderlong2002}.

\subsection{Formative essay}\label{sec:formative}

\subsection{Written resources}\label{sec:blog}

Non-native students
