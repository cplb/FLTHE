\chapter{Peer observations \& feedback}\label{ap:peer}

\section{Overview}

Peer feedback is valuable as it is from the point of view of another teacher, and so draws upon expertise regarding both the theory and practise of teaching; it is thus complementary to student feedback (see \apref{student}), which is more focussed on individual learning experiences. Feedback from others can give new insight into practices and an external perspective on techniques. In this appendix are presented two forms different types of feedback, the first, in \secref{ALS}, is from a discussion during an action learning set, and the second is peer observation.

Peer observations are a valuable tool for improving performance \citep{Ketteridge2015}. These external observations are useful input for self-reflection, providing counterbalance to it becoming excessive omphaloskepsis. Feedback on my teaching is presented and discussed in \secref{me}.\footnote{The structure of the feedback is based around University of Birmingham peer observation forms \url{https://intranet.birmingham.ac.uk/as/registry/legislation/codesofpractice/peerobservation.aspx}.} Observation of the teaching of others also provides a learning experience, as it can introduce new approaches and highlights areas for evaluation in one's own teaching. My feedback on observing the teaching of a peer is presented in \secref{other}.

\section{Action learning set discussion}\label{sec:ALS}

As a component of this course, action learning sets present the opportunity to discuss issues with other teaching academics from a range of disciplines. Topics can range from dealing with colleagues to problems of globalisation \citep[cf.]{Marshall2015}. The notes in \secref{ALS-notes} here are based upon discussions from 23 October 2014. These helped me cement my plans for writing a blog post on essay writing; I had already decided to set a formative essay, but these discussions reinforced that this should be a useful experience. Present were Mark Laver, Agnieszka Leszczynski, Sofia Malamatidou, Jennifer Marshall, John Snaith and Ying Zhou. These notes are expanded and evaluated in \secref{ALS-conclusion}.

\subsection{Discussion minutes}\label{sec:ALS-notes}

\subsubsection{Issue: Communication skills \& essay writing}

Major factors:
\begin{itemize}
\item Outside range of experience.
\item Time constraints---need to practise.
\item Poor command of language (especially for non-native speakers).
\end{itemize}

Points of reflection:
\begin{itemize}
\item Students do not know what is expected until shown.
\item Citations are not intuitive.
\item Make allowances for language problems.
\item Motivation---might not take interest in skills.
\end{itemize}

Agree action plan:
\begin{itemize}
\item Give good and bad examples for discussion (peer marking).
\item Guide examples of referencing.
\item Make sure to provide formative feedback for review.
\item Write up points for students to read outside of class.
\end{itemize}

\subsection{Discussion summary}\label{sec:ALS-conclusion}

The discussion focussed on the communication skills of students and essay writing. These skills need to be developed over time. Physics students, who do not frequently have to write essays, lack experience. They may be unaware of mistakes they are making or how they are to approach tasks such as referencing, which was highlighted as both important and difficult to master (see \ldots). Student performance should improved through practice; however, this requires both the opportunity, and the time and inclination to take advantage of this.\footnote{Motivation \ldots} A formative essay (and the accompanying feedback) should therefore be useful. I would be giving my students feedback on their practise essays in the following tutorial (tutorial 3, see \apref{plan}), and this reinforced that it is not enough that students gain experience through producing the work, they must also get detailed feedback on this. Examples of how to approach difficult or unfamiliar tasks could aid their understanding. A written reference such as a blog post covering these topics would be useful as students can refer back to this and study it in their own time. I would produce this ahead of the tutorial (see \ldots). Additionally, presenting examples for discussion in tutorial, can help students understand particular aspects of good and bad style. These can be discussed or perhaps marked by the students, so that students can get a better appreciation of these aspects \citep[chapter 10]{Ramsden1992}. I included a discussion of examples in tutorials; I asked students for their opinions and to point out good and bad features, but I did not actually give them anything to mark. This could be an activity to include in the future (\ldots). Overall, the discussion gave positive feedback on my ideas for teaching communication skills and highlighted particular aspects that could be enhanced or might be particularly useful.


\section{Observations of my teaching}\label{sec:me}

I was observing teaching one of my regular tutorials on 30 October 2014 (tutorial 3, see \apref{plan}). In this tutorial I gave feedback on practise essays, making it particularly pertinent to my teaching of essay writing, as well as covering the usual problem sheets and lecture material. My group was made of three students (the fourth was absent for personal reasons). There were two observers, both are scientists who give SGTs, but they are from different schools within the University. Observer B is from the School of Physics \& Astronomy, and I observe their teaching in \secref{other}. Comments from the two observers in \secref{me-A} and \secref{me-B} respectively, and my own reflections together with a discussion of the observations are given in \secref{me-discuss}.

\subsection{Observation A}\label{sec:me-A}

\subsubsection{Preparation \& planning}\label{sec:A-preparation}

Christopher had clearly planned and thought through the session that he wanted to deliver---all their work had been marked and annotated thoroughly and from this it was apparent that he derived the key points that needed to be covered in the session. The marking of the essay assignment was colour coded and from where I was sitting looked very thorough. In addition Christopher had made a webpage full of information for the students on how to write essays that they could read in their own time.

\subsubsection{Start of the session}\label{sec:A-start}

Christopher talked to the students prior to the start of the session and clearly has a genuine interest in them. The essay and question assignments were handed back and how they had been marked was clearly explained. Due to the nature of the session (going over problems that had been set previously) the students knew the format and what was expected of them. They seemed comfortable.

\subsubsection{Explanation of the topic or session material}\label{sec:A-topic}

I thought that the explanations to the students were really great (I have been aiming for better pace and legibility when explaining things since the observation). Even though I didn't grasp fully all the mathematical concepts that were being discussed I could understand enough to follow the logical delivery and the explanations used. I especially liked the use examples and parallels from everyday life that he used.

\subsubsection{Presentation of the session}\label{sec:A-presentation}

The tutorial was mostly based on a ``chalk and talk'' format going over key points from the questions the students had answered but to say that oversimplifies. Within the confines of the topic Christopher engaged the students and checked their learning.

When it came to discussing the essays each of the students were mentioned to give examples of when things had been done well. They were given a recap over what was expected in an essay to build on/explain the feedback they had already received in the marking of the essay.

\subsubsection{Student participation}\label{sec:A-participation}

The students seemed happy to ask questions about the topics and speak up when there were points they were unsure of. As the students had already answered questions and fulfilled an essay assignment prior to coming to the session they were already engaged in the process and I think that was very helpful.

\subsubsection{Finishing the session}\label{sec:A-finish}

Christopher finished by giving a good summary and discussed what would happen in the next SGT.

\subsubsection{Strengths}\label{sec:A-strengths}

I thought that your interaction with the students was good and the way that you explained the mathematics---unhurriedly and simply was excellent. I thought that the way you walked them through writing an essay was also very helpful.

Your dedication to helping them was apparent throughout the tutorial.

\subsubsection{Issues to be addressed}

I can't really think of any. It appeared to be very good from where I was sitting! 

\subsubsection{Areas of good practice identified}\label{sec:A-good}

You clearly had a good relationship with your students and work on it (you clearly want them to get the jobs/positions they want at the end of their studies and will help them in their aims). 

I especially liked your pacing when you worked through the questions the students had set. Your picking out of key themes and the way you walked them through those points was excellent. That you encouraged the students to ask questions about things they didn't understand, and that they did, is really good. All the students appeared to be engaged and interested.

I thought that your explanations were very good and helpful. When you talked about the essay they wrote I thought it was good that you identified and brought out the areas that they had all done well. 

\subsubsection{Innovative methods}\label{sec:A-innovative}

Great use of the internet and your blog to give your students advice on writing essays. I have stolen the link to give to my students! 

\subsubsection{Areas of general concern}\label{sec:A-concern}

I don't really have any areas of concern.

My only thought was that one of your students was very quiet. But, as you test how they're all getting on every week, I know that you would be able to directly or indirectly address any problems they might be having anyway!

I'm never sure with small groups whether it's good to have a handful (as you do) or to have slightly more. On one hand in groups of 3--4 you can build much more meaningful relationships with your students and tailor your teaching to each of them. However, small groups are also rather exposing and don't provide the assurance that larger groups may provide (it's comforting to know that you're not the only one that doesn't ``get in'' in a large group)---not that you can tailor learning in a large group! 

The format of your SGT was very different to the tutorials that I have run in the medical school. Thinking of ways to improve the content that you had to deliver is very difficult. My only thought would be making the students work through an example question together in the tutorial. But I have no idea how long that would take, or how beneficial that would be when the point of the tutorial is to help them with questions they have already answered.

\subsubsection{Additional comments}

Like I said (see other feedback), this was a very different experience for me and I've tried to incorporate some of the things you do very well into my tutorials since. Your calm demeanour is especially something I would like to possess!


\subsection{Observation B}\label{sec:me-B}

\subsubsection{Preparation \& planning}\label{sec:B-plan}

Tutorial appeared to be well planned, consisting of a chalk and talk exposition of problems encountered by students in their Problem Sheets, followed by a thorough look at essay writing techniques. The latter was not rushed, so evidently the timing of the individual parts of the tutorial had been well thought through.

\subsubsection{Start of the session}\label{sec:B-start}

The tutorial began with a chalk and talk exposition of Problem Sheet issues. Here Christopher seemed a little nervous, and the students rather quiet, so it was not clear if they were following the explanations. This may have been due in part to the presence of two observers, though one could ask a few questions now and again of the students to check they're keeping up.

\subsubsection{Explanation of the topic or session material}\label{sec:B-topic}

Good use of the blackboard for explanations of Problem Sheet problems. A few small technical opportunities were missed e.g. introduction of metric tensor, and perhaps more graphs e.g. for probability distribution functions might have helped, however the oral explanations were crystal clear and self-contained. Christopher appeared to become progressively more relaxed through these explanations, which was good.

\subsubsection{Presentation of the session}\label{sec:B-presentation}

The session clearly had two parts. The first part felt a bit haphazard and the transitions between topics were very quick, though this is not unexpected given the breadth spanned by the different Problem Sheets. If possible, it could be useful to find/draw out a common theme here, would help students to see the bigger picture.

The second part, on essay technique, consisted of handing back and giving feedback on preparatory essays that the students had written, followed by a look at technique in general. This was very useful and students asked a few questions here so they were evidently engaged. Armed with the background that Christopher has clearly put a lot of thought into putting together, no doubt the essays from these students will be among the best in the School. Setting preparatory essays is a good idea, as long as such extra works, in addition to Problem Sheets/Skills tests etc., does not inadvertently overload the students.

\subsubsection{Student participation}\label{sec:B-participation}

This aspect could do with some attention, bearing in mind that Personal Tutors are one of the few people (the others being the Welfare Tutor and Senior Tutor) that students should feel comfortable approaching if e.g. they have personal issues (which can then be referred on to the Welfare Tutor if needed). The students were extremely quiet during the first part of the tutorial. During the second part of the tutorial, a few (but not all) of the students provided some questions.

\subsubsection{Finishing the session}\label{sec:B-finish}

Following the essay-writing workshop, Christopher concluded the session by asking the students if there were any questions in general, which was good, and a few students used this opportunity well.

\subsubsection{Strengths}

Preparation was clearly superb, and structure of tutorial well thought out. Chalk and talk explanations of technical problems were excellent. Christopher did the entire tutorial without reference to notes, which was very impressive.

\subsubsection{Issues to be addressed}\label{sec:B-issues}

Student interaction and participation: suggest spending 5--10 minutes at the start to ask students how they're getting along, which Lectures/Problem Sheets they found particularly difficult/easy, whether they've thought about talk/essay titles, whether they've seen anything Physics-related in the news etc.; one or two of these would help to break down any communication barriers and hopefully all the students might feel more comfortable interacting around more difficult topics later in the session. Then, while highlighting particular issues with the Problem Sheets, one could ask questions of the students to see if they're keeping up: these questions don't have to be difficult and could even be multiple choice via a voting system.

\subsubsection{Areas of good practice identified}

Tutorial and ancillary aspects w.r.t.\ essay writing were extremely well prepared, with a well thought-out structure. Technical explanations were clear.

\subsubsection{Innovative methods}

Preparatory essays were set as a formative exercise before the summative essay assignment in the spring term. An online blog was also created, consisting of a useful guide to good essay writing technique.

\subsubsection{Areas of general concern}\label{sec:B-concern}

Interaction with students could be improved. It was not clear in some parts of the tutorial that the students were following.

\subsubsection{Additional comments}

Overall a good tutorial with technical problems and essay technique covered particularly well. Student interaction could be improved.


\subsection{Discussion}\label{sec:me-discuss}

\subsubsection{Reflections}

This was a difficult tutorial. There were two distinct learning objective, the first being the usual covering of material from recent lectures and problems sheets, and the second concerning essay writing. Covering both of these within an hour is challenging, and the situation is further aggravated by having not had a tutorial the week before (see \apref{plan}). I decided to defer discussion of the mathematics problem sheet until the next week (as there were no serious problems), but I did not want to postpone giving feedback on anything else. This is both because feedback is more beneficial when prompt \citep[chapter 4]{Gibbs2015,Jaques2007} and so that concepts that be solidified in time for use in future work.\footnote{An example of an important topic from this week's tutorial was relativistic mechanics, which would appear both in the Classical Mechanics course and the Particle Physics course. It is in this context that Observer B mentions the metric (\secref{B-topic}); this is an idea that I would try to explain at later point as not all of my students had sufficient matrix-algebra skills at the time.} Given the amount of material to discuss, I was concerned that I would not be able to time things correctly. This anxiety is probably what Observer B picks up on in \secref{B-start}; it is something that I should be conscious of as non-verbal communication can influence learning \citep[chapter 2]{Brown1988}. However, I did succeed in covering the material and did so without rushing over the material (cf.\ sections \ref{sec:A-strengths}, \ref{sec:A-good} and \ref{sec:B-plan}).

The breadth of topics covered within the session could have made the material difficult to assimilate. Observer B noted rapid transitions between topics (\secref{B-presentation}). I do not believe that this is necessarily a downside, as a change in topic can renew attention \citep[chapter 2]{Brown1988}. However, this makes it necessary to remind students of all the items discussed, lest some should be forgotten. I tried to incorporate a summary at the end of the session to do so (cf.\ \secref{A-finish}), but I think that this is something to work on in the future---it may be beneficial to ask the students to recall what we have discussed to help it become solidified in long-term memory \citep[chapter 2]{Brown1988}.

Student participation is a central tenet of SGTs. The ability to discuss ideas and ask questions is useful to students, and the opportunity to interact with students gives the tutor opportunity to check students' progress \citep[cf.][chapter 6]{Jaques2007}. This tutorial was something of an anomaly in that its emphasis was on providing feedback and lecturing on good essay-writing technique instead of explaining concepts and solving problems. However, doing more to engage the students could help them to better absorb that material presented. Both observers mention the possibility of setting problems (e.g. sections \ref{sec:A-concern} and \ref{sec:B-start}), this is something that I normally incorporate into regular tutorials, but it could also be something to add in the context of essay-writing. I had not previously considered this, but working through examples of good or bad style might engage students and help their understanding.

The engagement of students is mentioned prominently by both observers. They mention that students seem comfortable (\secref{A-start}), interested (\secref{A-good}) and engaged (\secref{B-presentation}), and are happy to ask questions (sections \ref{sec:A-participation} and \ref{sec:B-finish}). However, there is a clear distinction between the two reports too. Observer A is generally positive with regards to student participation, noting how they are happy to speak up (\secref{A-participation}) and that I both took an interest in them (\secref{A-start}) and their understanding (\secref{A-presentation}). In contrast, Observer B expresses concern that students are too quiet (\secref{B-participation}) and may not be following (\secref{B-concern}). I think that this apparent contradiction can be explained as Observer B arrived late. They therefore missed the first few minutes of tutorial where I chatted with the students about their work, progress and other topic (exactly as suggested by Observer B in \secref{B-issues}). Having not seen me establish this rapport, they arrived when the students had settled in and their attention could be mistaken for shyness. It may be that both observers established their opinions on student participation in their first few minutes, and subsequently reinforced these opinions in an example of confirmation bias \citep{Ross1975,Lord1979}.\footnote{One could conclude that this is an inevitable weakness in peer observation; however, this might just reflect my own predisposition as a physicist against qualitative data.} The difference in opinion potentially indicates that the students could have significantly different opinions on the dynamic of a tutorial too, particularly if they arrive late and miss the introductory first few minutes. In \apref{student}, we see that the students are generally comfortable in tutorials.

Turning to essay writing, feedback is generally good. My marking was praised as detailed (\secref{A-preparation}), which is consistent with feedback from the students (\ldots). I was concerned that the detailed marking could be viewed as overly critical, hence I tried to also highlight good aspects of each piece of work to offset this  (\secref{A-good}). The writing of a blog post was highlighted by both observers (see \secref{sec:A-innovative} and also \secref{sec:other-discuss}). I am happy with how the session went, the students took an interest in my explanations of how to compose a good essay. I feel that this was stimulated by having just written one themselves. In conclusion, Observer B was extremely enthusiastic about the essay preparation, predicting that essay from my students would be amongst the best in the year (\secref{B-presentation}); this was overly optimistic (\apref{marks}), as good teaching does not necessarily lead to successful learning \citep[chapter 5]{Ramsden1992}; however, I believe that this session did succeed in its aim of teaching some good practices in essay composition.


\subsubsection{Conclusions}\label{sec:me-conclusions}

Overall, both observations are positive. My ability to explain concepts was praised by both (sections \ref{sec:A-topic} and \ref{sec:B-topic}). Different opinions on student participation were taken by both, but I think that these stances can be reconciled by considering the Observer B did not experience the complete tutorial. Although I am not concerned by a lack of student participation, it is something which can be improved, and I shall try to do so. The key concepts I take away from the observation are:
\begin{itemize}
\item Student engagement is good, but could be improved. Making further effort to include problems for the students could be beneficial. Asking students to vote, rather than individually answering a question could help involve more shy students.
\item Explanations are clear and well paced. Hence I can be more reassured in my abilities.
\item The creation of a blog entry on essay writing is viewed as a valuable resource.
\end{itemize}
I am happy with how the tutorial went, and that my teaching is of an acceptable standard.

\section{Observing the teaching of others}\label{sec:other}

I made two observations of the teaching of Observer B, who is also teaching within the School of Physics \& Astronomy. I observed both a lecture delivered (as part of a course) to third years (a class of 18) and a tutorial to second years (a group of three, of which two arrived later) on 21 November 2014. The latter is directly comparable to my own teaching; the former is less so, but there are many transferable skills between lecturing and tutoring (see \secref{teach-n-learn}). This was especially true in this instance, where a particular effort was made to interact with the class. I was also asked to consider how the class could be encouraged to do the optional examples questions that accompany the lecture course, a problem similar to asking students to complete additional work for tutorials. My notes on the lecture and tutorial are recorded in \secref{other-lecture} and \secref{other-tutorial} respectively, and my reflections on this are discussed in \secref{other-discuss}.

\subsection{Lecture observation}\label{sec:other-lecture}

\subsubsection{State of venue and suitability for lecture}

Venue is adequate for class size. Lack of clock makes timekeeping difficult.

\subsubsection{Strengths of sessions observed}

Communication with class: good rapport established, students happy to ask questions. Students remained attentive throughout.

\subsubsection{Weaknesses of the session observed}

Slightly rushed at end: no time for summary (not to much of a problem as course on-going). Potentially lost some students during difficult step in derivation---it took a while for students to catch-up (but they did).

\subsubsection{General comments}

Some slides cluttered. Lots of talking to the board. Handled computer problem well. Students are given handout.

\subsubsection{Actions recommended}\label{sec:other-lecture-actions}

Check students are happy with key steps before moving on. Encourage students to do examples sheet by introducing competitive element of progress feedback; results of problems could be made use of in lectures.

\subsubsection{Areas of good practice identified}\label{sec:other-lecture-practice}

Excellent communication with students: strong rapport has been established and students are happy to ask questions. Students involved through question asking (either for verbal or hand-up responses). Good use of the board for derivation. Pacing generally good, but could be improved by checking with class before moving on. Students remained attentive throughout (a probable consequence of the two above points).

\subsubsection{Innovative methods}

Nice use of numerical example in demonstrating tunnelling probabilities.

Good use of interactivity: this helps to keep students engaged, helps to check that they are happy with concepts, and provide an environment where students are comfortable asking questions.

\subsubsection{Areas of general concern}

Try to improve timekeeping to ensure time for summary at the end. This can emphasise the main points and provide a chance to look forward to the next lecture.

Try not to talk too much to the board.


\subsection{Tutorial observation}\label{sec:other-tutorial}

\subsubsection{Preparation \& planning}

Showed good familiarity with problem sheets (was able to offer guidance on tackling problems).

It would be useful to be more familiar with the errors made by students in order to give specific feedback and tailor material covered.

\subsubsection{Start of the session}\label{sec:other-tutorial-start}

Started well: checked well-being of students and status of their work.

Asked about progress with presentations and essays. This might help motivate students who have not made as much progress. Combining the two might also help students make connections regarding skills required. It is useful for students to start thinking about essay early, as encouraged here.

Had to contend with students arriving late.

\subsubsection{Explanation of the topic or session material}

Covered a selection of problem sheets (mostly particle physics), this was matched to the students' needs. Explanations were fine, and included some examples.

\subsubsection{Presentation of the session}

Standard board-based presentation, which is fine and well-suited to the material. Pacing is good, students can follow but do not get bored.

\subsubsection{Student participation}\label{sec:other-participation}

Good: students are happy to ask questions and say when they do not understand.

There is potentially an issue with a single student dominating the session. It may be worth asking students in turn (or similar).

\subsubsection{Finishing the session}

Kept good time.

It might be a good idea to confirm time and place of next tutorial (if possible) to help avoid confusion.

\subsubsection{Strengths}\label{sec:other-tutorial-strengths}

Good conversational style. This helped students feel comfortable, which is important for them to ask questions. This also allows for students to admit that they have not done work (i.e. start on presentations), which can allow for support to be offered.

Checked student well-being and encouraged students to manage work-load efficiently (not leaving presentations and essays until the last moment).

\subsubsection{Issues to be addressed}

Lack of regular time and location leads to confusion and loss of contact time. Extra reminders are needed.

\subsubsection{Areas of good practice identified}

Good conversation style and rapport with students. Shows interest in students asking about welfare and workload, which helps them to think about their own time management.

Good use of board to work through problems. Used current problems to revise material previously covered: good for revision and establishing connections between areas. Especially beneficial for maths!

\subsubsection{Innovative methods}\label{sec:other-tutorial-innovative}

Using progress of students to help motivate others in the group (for example, that one student had already finished their presentation). Most teaching techniques tried-and-testing.

\subsubsection{Areas of general concern}\label{sec:other-tutorial-concern}

Students confused about time and location of sessions. This is inevitable when timetabling does not allow for a consistent slot. Students are responsible for organising themselves, but perhaps extra reminders or a simple means for them to look up their next session would prevent them missing out.

\subsubsection{Additional comments}

Board might not always be visible to students: it might be useful to move chairs around.

\subsection{Discussion}\label{sec:other-discuss}

These were two good examples of teaching. In both cases, good communication is established with the students, such that they are both happy to be asked questions and are happy to ask questions (e.g., sections \ref{sec:other-lecture-practice} and \ref{sec:other-participation}). This can be difficult to achieve in lectures, where interaction is less traditional. The comparatively small class size most probably helps with this. The interaction helps both to understand how the students are progressing, and to ensure that they can guide their learning on difficult topics. The engagement in lectures does not seem to transmit to doing the unassessed problems as the majority of the students had not kept up with these. This demonstrates that an interest in a topic does not necessarily translate into motivation. It is also potentially an example where there is the cultural norm is to not do the problems, giving an additional social incentive not to do the work.\footnote{Completing the problems could be encouraged by setting higher expectations for the students, perhaps be setting them a target score or by running some optional lectures or examples classes where only students who attempt the work may attend \citep[cf.][chapter 10, passports for seminars case study]{Jaques2007}} In the tutorial, a similar social pressure was used to encourage study, as the progress of one student was used as an example to the others (sections \ref{sec:other-tutorial-start} and \ref{sec:other-tutorial-innovative}). This seemed effective, but could be counter-productive in some cases, should the students be made to feel uncomfortable (either due to pressure or praise). Knowing your students allows them to be taught effectively, and this seems to be the case here (e.g., \secref{other-tutorial-strengths}).

The teaching methods used were traditional, following a similar framework to that I use myself. This is reassuring, but not surprising. Standard practises, such as going through problems on the board in a tutorial, have evolved because they are effective and efficient. While this means that I cannot acquire new teaching techniques form these observations, I can still learn from how they are implemented. The good rapport established with the students highlights how important it is for a tutorial (or a lecture) to not just be a monologue from the teacher, this had made more conscious of how much time I spend speaking and how I can encourage participation. Asking questions is the simplest means of involving students (e.g., \secref{other-lecture-practice}). These can vary from simple hands-up votes (to include shy students, or prevent domination from individual students) to longer open-answer questions; asking students' opinions on responses from their peers could promote discussion between group members instead of just between the teacher and student \citep[cf.]{Foster1981}. 

There was one particular area of concern from the tutorial observation, students arrived late to the session (\secref{other-tutorial-concern}). Only one student arrived on time, the other two arrived as the session progressed. Missing the tutorial is suboptimal as they miss out on teaching time. It could also effect their perception of tutorial, as discussed in \secref{me-discuss}. The lack of punctuality could be attributed to the students not valuing tutorial time (\ldots); however, the simplest explanation is that they are confused by their timetable. These tutorials were not regularly scheduled, moving in both time and location. Sticking to a routine, which I try do, helps students to schedule their time and prevents these mishaps.\footnote{I did have to reschedule tutorial 6 (\apref{plan}) from my usual time due to an unfortunate clash with \emph{this} course.}

In conclusion, these observations have reinforced my conclusions from \secref{me-conclusions},
\begin{itemize}
\item Student participation is important. Although my students are engaged in tutorial, this is something I can improve.
\item My approach to teaching is appropriate.
\item My blog on essay-writing is a useful resource---it was recommended to Observer's B students during the tutorial.
\end{itemize}
This is useful information, that can be used in improving my teaching.
