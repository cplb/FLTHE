\chapter{Evaluation \& conclusions}\label{ch:conc}

\section{Reflections on 2014/2015}

Went well
More time good
Connection to talks good

Blog useful (cite peer)
Attendance

Difficult to evaluate effectiveness
Clear examples of improvement
Not clear if extra work necessary

Students did not get maximal gain
Feedback more useful

\section{Ideas for 2015/2016}\label{sec:future}

Evolution of ideas (no revolution)
Further integration of communication skills
Extra on feedback (at start)

However, bigger punch in Spring

Longer for formative assessment
Peer-writing/examples

Further blog


Second-year students more concerned with learning, with developing the ability to understand and to produce high-quality work, and having access to the support to enable them to achieve these goals \citep{Zaitseva2013}. Students more strategic in their working. Become more likely to adopt a performance rather than a mastery goal \citep{Lieberman2007}. Are less interesting in their subject and become less motivated to learn \citep{Lieberman2007}.

"Exploration of 'feedback'-related comments revealed that delays or the
inability of feedback to guide further learning were criticised by the
second-year students. It also appeared from the comments that second-year
students were more concerned with feedback as a means for improvement
and progression, rather than encouragement and affirmations, in comparison
with first-year students." \citep{Zaitseva2013}

Engage students in feedback and setting their own goals to become self-regulating learners \citep{Nicol2006}

Ask students to work together as a formative excercise, engaging in a peer-learning activity such as collaborative writing, peer editing or peer marking \citep[chapter 1]{Falchikov2001}.

Difficult to evaluate the effectiveness of feedback \citep{Price2010}

\section{Summary}

Mastering communications skills is important both for establishing competency as a physcisit and for emploability. Learning to produce scientific writing requires understanding how more familiar, less specialised formal writing must be specialised, various conventions for style and presentation, and troublesome concepts like referencing. Covering all of these withing the constraints of tutorials is challenging. A written resource that can be reviewed later is useful. A formative assessment could be useful is studnets can make use of the feedback from it. Ensuring students can digest feedback is an area where my teaching could be improved.
