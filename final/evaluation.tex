\chapter{Evaluation \& conclusions}\label{ch:conc}

\section{Reflections on 2014/2015}

Overall I am happy with my teaching this year. I received good feedback on tutorials from both my students (\secref{tutor-results}) and from peer observation (\secref{me}). More specifically, I think that the teaching of communication skills went well, and that spending more time discussing communication skills was beneficial (see \secref{comm-time}). Integrating the different aspects of communication skills seems to have been effective (\secref{talk-essay}) and established a connection between different aspects of tutorial teaching.

Creating blog posts was a useful exercise (sections \ref{sec:views-blog} and \ref{sec:me}) and one that can be reused for later years. Written resources are especially useful as they can be used by students even if they cannot attend tutorials, which helps to ameliorate the negative impact of missing teaching (\secref{plots}).

It is difficult to evaluate the effectiveness of feedback \citep{Price2010}, accordingly it is difficult to assess to impact of the formative essay. There is no significant improvement in the average attainment of my students compared to last year, as discussed in \apref{marks}, and students varied widely in their assessment of its benefits (\secref{views-form}). It is therefore unclear if it is worth the extra work.

From the student evaluation, it appears that they are not extracting the full value from feedback (\secref{views-essay}). Making the feedback easier to absorb may help. Additionally, making the formative essay more closely resemble the final one (requiring a scientific subject), could help students see the parallels (\secref{views-form}).

My teaching appears to have been more consistent this year, with fewer students under-performing in the essay (\secref{sec:plots}), but it also saw fewer students performing exceptionally well. While this could be an artefact of a small sample size, I would like to capture the best aspects of my teaching from the last two years.

\section{Ideas for 2015/2016}\label{sec:future}

Building this year, there are improvements that can be made. These are an evolution, rather than a revolution, of my current approach. A sketch tutorial plan is given in \tabref{2015-16}.

The first change is further integration of communication skills. In the first tutorial I introduce the presentation and the formative essay to link them from the start.

At the same time I will introduce tutorials, highlighting the inclusion of communication skills and feedback from problem sets.\footnote{Since communications skills and the completion of problem sets are highly valued by students (sections \ref{sec:comm-value} and \ref{sec:tutorial-value} respectively), this may help them appreciate the importance of tutorials.} In second year, students become more concerning with producing high-quality work and having the support to achieve this \citep{Zaitseva2013}, and less motivated to master their subject \citep{Lieberman2007}. Discussing how to use feedback (and tutorials) may guide their practise and help me to produce useful comments.

I will give two weeks for the practise essay, slightly longer than this year. To make the practice essay more similar to the final one, I will suggest that they rewrite their first-year essays. This should ensure they are not overburdened and encourage them to use of past feedback.\footnote{Second-year students are more concerned with feedback as a means for improvement and progression than first-years \citep{Zaitseva2013}, so this may help their transition into the second-year mindset.}

When giving discussing the formative essay in tutorial 4, I will not go into as much detail. Topics will be stripped to a minimum to help students focus.\footnote{The identification of threshold concepts (\apref{threshold}) is useful for picked these topics.} Time will instead be used on a peer-learning activity, such as collaborative writing or editing and peer marking \citep[chapter 1]{Falchikov2001}, an idea supported from discussions with my peers (\secref{ALS}) and my personal development (\apref{Scriptoria}). This should reinforce ideas from discussion. Following this, I will ask students to reflect on their work and set goals for the final essay, encouraging them to self-regulate their learning \citep{Nicol2006}.

Less troublesome concepts and those that are less transferable to the oral presentation, will be discussed in the Autumn term. Discussing this while the essays are being written may make them more relevant, and give a larger impact to the lesson (\apref{marks}).

I shall produce further blog posts after consulting with the students. These posts could include examples that can be marked by the students.

\section{Summary}

Mastering communications skills is important both for establishing competency as a physicist and for employability. Learning to produce scientific writing requires understanding how more familiar, less specialised formal writing must be specialised, various conventions for style and presentation, and troublesome concepts like referencing. Covering all of these within the constraints of tutorials is challenging. A written resource that can be reviewed later is useful. A formative assessment could be useful if students can make use of the feedback from it. Ensuring students can digest feedback is an area where my teaching could be improved.
