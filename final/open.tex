\chapter{Overview}

In this work we examine the small-group teaching of second-year physics undergraduates. This teaching takes the form of regular tutorials. The role of small-group tutorials (SGTs) within physics is introduced in \chapref{intro}.\footnote{In the context of teaching physics, we consider a small group to be one consisting of 2--6 students, which follows \citet[chapter 1]{Jaques2007}. My own groups range in size between 2 and 5 students.} Here, I reflect upon my own educational experiences as well as research on teaching and learning \citep{McCarthy2008}. One of the components of tutorials is the teaching of key skills: compiling a curriculum vit\ae{} (CV), giving a presentation and writing an essay. These transferable skills are valuable assets when the students graduate and seek employment \citep{Pike2015}. Teaching these communication skills is different from the more familiar teaching of subject material and so I have chosen to examine my practices in more detail, concentrating on essay writing. In \chapref{essay} I discuss a plan for this teaching, I introduce a new elements to my delivery, a formative essay designed to give feedback \citep[feedforward;][]{Bloxham2015} and practice, and blog posts that can be referred to outside of tutorial.\footnote{The posts can be read at \url{http://cplberry.com/tag/writing/}.} In \chapref{conc}, I evaluate their impact. In the process of this, I draw upon feedback from students (\apref{student}) and peers (\apref{peer}), my own training (e.g., \apref{Scriptoria}), and the final marks (\apref{marks}). I find that the formative essay has a small positive effect on the achievement of my students, but this impact could potentially be improved by adapting its format; as might be expected, the students' motivation and proficiency remain the dominant predictors of their essay-writing ability \citep{Ketteridge2015}.
