\chapter{Threshold concepts}\label{ap:threshold}

Threshold concepts are core concepts in a discipline that are instrumental to mastery of the subject, but troublesome to learn \citep{Meyer2003,Meyer2005}. They limit development until they are understood, when the student may progress and expand their range of competences. They have several distinctive characteristics; they are \citep{Mathieson2015}
\begin{itemize}
\item Transformative---Once understood they change how the students views the subject or a larger aspect or their lives \citep[cf.][chapter 2]{Kolb1984}.
\item Irreversible---Once learnt there may be no changing back to a state of innocence.
\item Integrative---Mastering a threshold concept can reveal interconnections between ideas and allow new understanding to be accessed.
\item Troublesome---Threshold concepts are difficult to assimilate. They may be conceptionally challenging or counter-intuitive; their transformative nature may require a student to reject previously held beliefs or their irreversiblity may be accompanied by a sense of loss, making the student resist their progression.
\end{itemize}
Identification of threshold concepts is difficult and benefits from a consensus of teachers and students \citep{Barradell2013}, hence I will only consider my identifications as provisional; however, determining these concepts has aided my teaching by highlighting areas for particular, patient attention.

In \secref{essay-aims}, I group subject material into the categories of specialisations, conventions and threshold concepts. All three can be challenging for the students, and mastery of them is important for achieving proficiency as a physicist \citep[cf.][]{Gonsalves2014a}. However, I do not consider the specialisations as threshold concepts as they are not transformative, they represent an evolution of familiar knowledge rather than a revolution; similarly, mastering conventions do not qualify as they do not open up a new outlook on the subject, they just give understanding about themselves.\footnote{This lack of integration could demotivate students to learn conventions as they do not open up further knowledge.}

For second-year physics undergraduates, I have found three threshold concepts with regards to their scientific writing (as explained in \secref{essay-aims}):
\begin{enumerate}
\item Not restricting themselves to prose---This addresses themes under the Clarity and Presentation mark-scheme categories (\tabref{mark-scheme}), and potentially impacts the Understanding mark-scheme too, as good diagrams and use of mathematics can demonstrate good comprehension of the material.
\item Referencing---The Referencing and Reading mark-scheme categories are dependant upon this. Proper formatting of bibliographic elements can fall under Presentation.
\item Being quantitative---This is an aspect of the Basic writing style and Clarity mark-scheme categories.
\end{enumerate}
Descriptions of how each of these satisfy the criteria for threshold concepts are given in tables \ref{tab:words}, \ref{tab:citations} and \ref{tab:numbers}.
\begin{table}\scriptsize
\centering
\begin{tabular}{c p{4.5in}}
\toprule
\multicolumn{1}{c}{Criterion} &  \multicolumn{1}{c}{Explanation of fulfilment} \\
\midrule 
\multirow{1}{*}{Transformative}	 & Students often consider essay writing to use the set of skills they developed in English classes and not those more frequently used to describe physics, such as equations, diagrams and graphs. After realising that scientific writing does not have to be solely prose, they realise that they can communicate efficiently ideas that are extremely difficult to articulate in words. This liberates them to tackle ideas that they might have previously avoided. Students can also then consider how best to convey information, rather than how to attempt to communicate within the confines of a given medium. \\
\multirow{1}{*}{Irreversible}	 & Having discovered a means of communicating more efficiently, there is a significant effort barrier to revert back to a prose-only approach. In effect, students gain access to a new language, one of mathematics and pictorial representation, that they are already fluent in, and this aids their communication. \\
\multirow{1}{*}{Integrative}	 & This brings together all the aspects of communicating ideas that students have encountered. Once they realise that there is no taboo in mixing media, in fact that this is encouraged, they begin to see that all forms of communication are just different facets of the same skill set. \\
\multirow{1}{*}{Troublesome}	 & This requires breaking down barriers between different compartments of knowledge from those learnt in English, mathematics and science \citep[cf.][chapter 7]{Kolb1984}, and forces students to re-evaluate how they communicate. This may potentially force them to reconsider their proficiency at communication. The change may also be resisted as learning how to produce high-quality figures and equations, and how to include these within a written narrative requires significant effort. \\
 \bottomrule
\end{tabular}
\caption{Description of how the first threshold concept, that scientific writing does not have to be restricted to pure prose, qualifies as a threshold concept.}\label{tab:words}
\end{table}
\begin{table}\scriptsize
\centering
\begin{tabular}{c p{4.5in}}
\toprule
\multicolumn{1}{c}{Criterion} &  \multicolumn{1}{c}{Explanation of fulfilment} \\
\midrule 
\multirow{1}{*}{Transformative}	 & Once it is established that acknowledging the work and ideas of others is central to good scholarship, and that some sources of information are more reliable than others, students have new criteria for evaluating texts. This can help them to understand the quality of their own work, but it also gives them a means of evaluating the work of others. This can lead them to be more sceptical of work that is not properly cited. Furthermore, this can change their view of their own work to be just part of a chain, one that draws upon previous work and will be used by others in the future. \\
\multirow{1}{*}{Irreversible}	 & Good referencing is a sign of good work (or rather, poor referencing indicates poor work), motivating the inclusion of referencing. Furthermore, following someone else's references is a useful means of discovering new information, and including references in your own work is a convenient way of noting which source contains which piece of information; the usefulness of referencing provides a good incentive to continue the practice. \\
\multirow{1}{*}{Integrative}	 & Including references links the ideas of research (discovering information) and communication (disseminating information). That the two are components of the whole, that the knowledge one is attempting to impart is connected to the knowledge one drew upon to construct it, is a central tenet of scholarship. Referencing also integrates the idea that scientific research and making discoveries is linked to the skills taught at the library, and that properly citing sources is as important in terms of scientific ethics as reporting all results of an experiment.  \\
\multirow{1}{*}{Troublesome}	 & This concept is challenging as students are used to being rewarded for what they know or have deduced. By including references they must relinquish credit to others who preceded them. Related to this, students can find it difficult to distinguish between what requires a citation and what does not: what qualifies as general knowledge that they may state with impunity and what must be backed up? If they are drawing heavily on a set of sources, should every sentence end with a citation? How do they distinguish what is their own thought? Students also find it difficult to determine what makes a good source to cite: if the information on Wikipedia is correct, why should they not cite it? Why should they go to the extra effort of reading the source for the Wikipedia article? Finally, including in-text citations, properly formatting bibliographic entries (which is different for each type of source), and tracking down reliable references (rather than the first link from  search results) is labour-intensive, providing an effort barrier. \\
 \bottomrule
\end{tabular}
\caption{Description of how the second threshold concept, of proper referencing, qualifies as a threshold concept \citep[cf.][]{Warner2011}.}\label{tab:citations}
\end{table}
\begin{table}\scriptsize
\centering
\begin{tabular}{c p{4.5in}}
\toprule
\multicolumn{1}{c}{Criterion} &  \multicolumn{1}{c}{Explanation of fulfilment} \\
\midrule 
\multirow{1}{*}{Transformative}	 & As explained in \tabref{words}, bringing together writing skills with those learnt in mathematics or science opens up new means of communicating. Furthermore, as explained in \tabref{citations}, accepting the idea that statements must be backed up with evidence (a citation or data), gives a new means of evaluating work of oneself and others. \\
\multirow{1}{*}{Irreversible}	 & Providing quantitative information efficiently communicates specific knowledge, making it a desirable skill. After becoming used to considering definite, precisely defined quantities or concepts, it can be difficult to revert back to conversing with nebulous terms. \\
\multirow{1}{*}{Integrative}	 & This unites analytical skills with communication skills (as in \tabref{words}), and connects elements of discovery (research, measurement or calculation) with communication (as in \tabref{citations}) \\
\multirow{1}{*}{Troublesome}	 & As with including equations of graphs in written work (\tabref{words}), accepting this idea means decompartmentalising skills learnt in different subject areas. Furthermore, there may be a sense of loss associated with the realisation that phrases commonly used in conversation, like "it was quite hot", are not appropriate for rigorous scientific writing, and a new style must be cultivated. However, the greatest difficultly may be that assembling quantitative information is difficult. It may require calculations, data analysis or trawling through the literature. A simple statement, such as stating the Sun is hot, can become a major research project to determine exactly how hot the Sun is and how we know this. Since time is finite, students must try to determine when including such information is useful, and when its usefulness is outweighed by the cost of producing it.\\
 \bottomrule
\end{tabular}
\caption{Description of how the third threshold concept, of providing quantitative information, qualifies as a threshold concept.}\label{tab:numbers}
\end{table}

I came to identify each of these by a different means. I was alerted to the first by its converse: students rarely include prose to explain their working or conclusions in problems sets. This was (repeatedly) pointed out to me by my supervisor when I was an undergraduate. The immiscible nature of written English and mathematics (or diagrams) persists when prose dominates over equations, potentially indicating that both the absence of words in problems sets and their exclusivity in essays are aspects of the same threshold concept. The problems of referencing were highlighted by students themselves, as this is the topic they ask most frequently about and admit to being most confused by and anxious about. The final threshold concept is something that I have noticed from marking the work of students; it shares aspects with the other two concepts, potentially highlighting how aspects of scientific writing could be further integrated.

Discovering means of making these concepts less troublesome to learn would improve the effectiveness of teaching essay writing.
