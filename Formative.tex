\documentclass[a4paper, 11pt, twoside]{article}
% Set up page layout
\usepackage[a4paper]{geometry}
\usepackage{fancyhdr}
% Set up hyphenation, etc.
\usepackage[british]{babel}
% Set up images
\usepackage{pstricks,pst-node,pst-text,pst-3d,pst-eucl}
\usepackage{psfrag}
\usepackage{import}
\usepackage{graphicx}
\usepackage{subfigure}
\usepackage[figuresright]{rotating}
%\usepackage{flafter}
% Set up fonts
\usepackage{charter}
\usepackage{amssymb,amstext,amsfonts} %% ... with default font
\usepackage{amsmath}
\usepackage{mathrsfs}
%\usepackage{txfonts}
% Colouring
\usepackage{xcolor}
\definecolor{vermilion}{rgb}{0.80,0.40,}
\definecolor{blueishgreen}{rgb}{0,0.60,0.50}
\definecolor{skyblue}{rgb}{0.35,0.70,0.90}

% Set up lists and tables
\usepackage{listings}
\usepackage{enumitem}
\setlist{itemsep=0pt}
\usepackage{booktabs, longtable, dcolumn}
\usepackage{array}
\usepackage{ragged2e}
\newcolumntype{P}[1]{>{\RaggedRight\hspace{0pt}}p{#1}}
\usepackage[labelfont=bf,labelsep=space,hypcap]{caption}
\renewcommand{\captionfont}{\small}
\usepackage[caption2]{ccaption}
%\usepackage{tikz}
%\usetikzlibrary{shapes,arrows}
% More fonts
\usepackage[T1]{fontenc}
\usepackage{lmodern}
\renewcommand{\sfdefault}{lmss}
\renewcommand{\ttdefault}{lmtt}
\usepackage{microtype}
\usepackage[latin9]{inputenc}
\usepackage[11pt]{moresize}
%\usepackage{fixmath}
\usepackage{ upgreek }
% Special symbols
\usepackage{braket}
\usepackage{slashed}
% Set up page indexing, bib fomatting, etc.
\usepackage{afterpage}
\usepackage{appendix}
%\usepackage{paralist}
%\usepackage[linktocpage,bookmarksnumbered,breaklinks,pdfhighlight=/P,citebordercolor=blueishgreen,urlbordercolor=skyblue,linkbordercolor=vermilion]{hyperref}
\usepackage[authoryear]{natbib}
\usepackage[linktocpage,bookmarksnumbered,pdfauthor={Christopher P.L. Berry},pdftitle={Exploring Gravity},pdfkeywords={Ph.D. Dissertation}]{hyperref}            
\usepackage{bookmark}
\usepackage{etoolbox}
% Only link on year
\makeatletter

% Patch case where name and year are separated by aysep
\patchcmd{\NAT@citex}
  {\@citea\NAT@hyper@{%
     \NAT@nmfmt{\NAT@nm}%
     \hyper@natlinkbreak{\NAT@aysep\NAT@spacechar}{\@citeb\@extra@b@citeb}%
     \NAT@date}}
  {\@citea\NAT@nmfmt{\NAT@nm}%
   \NAT@aysep\NAT@spacechar\NAT@hyper@{\NAT@date}}{}{}

% Patch case where name and year are separated by opening bracket
\patchcmd{\NAT@citex}
  {\@citea\NAT@hyper@{%
     \NAT@nmfmt{\NAT@nm}%
     \hyper@natlinkbreak{\NAT@spacechar\NAT@@open\if*#1*\else#1\NAT@spacechar\fi}%
       {\@citeb\@extra@b@citeb}%
     \NAT@date}}
  {\@citea\NAT@nmfmt{\NAT@nm}%
   \NAT@spacechar\NAT@@open\if*#1*\else#1\NAT@spacechar\fi\NAT@hyper@{\NAT@date}}
  {}{}

\makeatother

\renewcommand{\bibfont}{\footnotesize}
\addto{\captionsbritish}{\renewcommand{\bibname}{{R\lowercase{eferences}}}}
%\renewcommand{\refname}{{R\lowercase{eferences}}}
\setlength{\bibhang}{0pt} 
%\setlength{\bibsep}{0.5pt} % 1 linestretch
\setlength{\bibsep}{0pt} % 1.3 linestretch
\addto{\captionsbritish}{\renewcommand{\contentsname}{{C\lowercase{ontents}}}}

%\usepackage{fourier-orns}
  
% Set page layout  
\headheight 13.6pt
% For 10pt text
%\headsep 0.6\baselineskip
%\footskip 2\baselineskip
%\textheight = 55\baselineskip
% For 11pt text
\topmargin -0.4cm
\headsep 0.6\baselineskip
\footskip 2\baselineskip
\textheight = 50.1\baselineskip

\def\today{\number\day\space\ifcase\month\or January\or February\or March\or April\or May\or June\or July\or August\or September\or October\or November\or December\fi\space\number\year}

% Custom commands
\newcommand{\eqnref}[1]{equation (\ref{eq:#1})}
\newcommand{\Eqnref}[1]{Equation (\ref{eq:#1})}
\newcommand{\secref}[1]{section \ref{sec:#1}}
\newcommand{\Secref}[1]{Section \ref{sec:#1}}
\newcommand{\chapref}[1]{chapter \ref{ch:#1}}
\newcommand{\Chapref}[1]{Chapter \ref{ch:#1}}
\newcommand{\partref}[1]{part \ref{pt:#1}}
\newcommand{\Partref}[1]{Part \ref{pt:#1}}
\newcommand{\apref}[1]{appendix \ref{ap:#1}}
\newcommand{\Apref}[1]{Appendix \ref{ap:#1}}
\newcommand{\tabref}[1]{table \ref{tab:#1}}
\newcommand{\Tabref}[1]{Table \ref{tab:#1}}
\newcommand{\figref}[1]{figure \ref{fig:#1}}
\newcommand{\Figref}[1]{Figure \ref{fig:#1}}

\newcommand{\units}[1]{\ensuremath{~\mathrm{#1}}}

\newcommand{\sub}[1]{\ensuremath{_\mathrm{#1}}}
\newcommand{\super}[1]{\ensuremath{^\mathrm{#1}}}

\begin{document}

\bibpunct{(}{)}{;}{a}{}{,\,}

\makeatletter
\pagestyle{fancy}
\renewcommand{\sectionmark}[1]{\markright{\thesection\ #1}}
\fancyhf{}
\fancyhead[LE]{Formative assessment}
\fancyhead[RE]{Christopher Berry}
\fancyhead[LO]{PCAP}
%\fancyhead[RO]{\rightmark}
\fancyhead[RO]{Teaching \& learning experiences}
\fancyfoot{}
\renewcommand{\footrulewidth}{0.1ex}
\fancyfoot[LO]{\today}
%\fancyfoot[LO]{21st August 2013}
\fancyfoot[RE]{University of Birmingham}
\fancyfoot[RO, LE]{\thepage}
\fancypagestyle{plain}{
\fancyhf{}
\renewcommand{\headrulewidth}{0pt}
\fancyfoot{}
\renewcommand{\footrulewidth}{0.1ex}
\fancyfoot[LO]{\today}
\fancyfoot[RE]{University of Birmingham}
\fancyfoot[RO, LE]{\thepage}}
\makeatother

\title{Formative assessment: Teaching \& learning experiences}
\author{Christopher Berry\\\href{mailto:cplb@star.sr.bham.ac.uk}{cplb@star.sr.bham.ac.uk}}
\date{\today} 

\maketitle

\vspace{-2\baselineskip}

\section{Introduction}

We learn from our own experiences, repeating things that have gone well, and avoiding those that led to failure \citep[chapter 2]{Skinner1954,Kolb1984}. My own teaching style reflects how I learnt as I attempt to emulate the experiences that were most educational for me. I read Natural Sciences at the University of Cambridge as an undergraduate, specialising in Experimental and Theoretical Physics, before continuing to complete a Ph.D.\ in Astronomy. During my postgraduate study I began supervising: tutoring small-groups (2--3 students), with teaching focused on weekly (formative) problem sets. Subsequently, I have moved to the University of Birmingham, where I now tutor: small-group (3--4 students) teaching similar to supervising, except problems also serve a summative role. At both Cambridge and Birmingham, I have taught second year physics. In this work, I discuss how my own learning experience have informed my teaching approach.

I begin by introducing core components of studying physics (\secref{physics}), before expanding upon small-group teaching (\secref{small}) which has been a primary component of both my learning and teaching experience. Building upon the ideas of learning in a small-group, I discuss my experiences of learning through peer collaboration (\secref{peer}). I conclude with a critique of how I have assimilated aspects of my own student experience into my approach to teaching.

\section{Studying physics}\label{sec:physics}

Physics is a broad subject covering the fundamental behaviour of the Universe and its components. It is familiar to most as it is introduced at school level: it's teaching follows a spiral curriculum, periodically returning to previously-taught topics to cover them in ever-expanding detail \citep{Bruner1960}. The main jump from school to university physics is in the use of mathematics. Applying mathematical formalism to describe and solve physical problems is the greatest threshold concept \citep{Meyer2003} faced by those wishing to become conversant with physics \citep{Wigner1960}.

My experience of university-level physics teaching has been traditional \citep[cf.][]{Iannone2015}. Concepts and theories are introduced in lectures. These ideas are reinforced through problem sets that also encourage mathematical fluency and independent learning \citep{Pike2015}. At least some of these problems are assessed to provide students feedforward information \citep{Bloxham2015}. Small-group teaching or examples classes give students the opportunity to discuss the problems and their solutions, as well as other elements of the syllabus.\footnote{This may fulfil the role of inverted classrooms, where students prepare outside of class and then work through problems under the guidance of a tutor in class, employed elsewhere \citep{Lage2000}.} Practical work is done in labs; experiments may complement topics covered in lectures, for example measuring refraction during a course on electromagnetism, be of equal importance as lectures for teaching ideas, for example in a course on electronics, or motivate material covered in lectures, for example error analysis. Assessment is dominated by summative closed-book examinations at the end of the courses.\footnote{The situation may be comparable to in mathematics where traditional closed-book examinations are favoured \citep{Iannone2014}, as both subjects require similar (mathematical) problem solving and the correctness of solutions is not subjective.} I have found completing problem sets to best encourage deep learning: good understanding is required to fully solve a question and, since problems are done outside of class, there is sufficient time to review assimilate new ideas.\footnote{Even if a student takes an achieving strategy \citep[chapter 2]{Biggs1987} and only invests time in understanding the topics directly probed by the given questions, a dense covering of the syllabus by problems sets should ensure a comprehensive understanding of the majority of the material.}

Problem sets are of paramount importance. Extracting the most from them relies on the student being motivated to put in the necessary effort: those who work harder shall learn more \citep{Gibbs2015}. This is difficult to externally inspire \citep[cf.][]{Ryan2000}. Consequently, we shall instead focus on how problem sets are supported, which is primarily through small-group teaching.

\section{Small-group teaching}\label{sec:small}

Supervisions at Cambridge involve either an academic or a PhD student meeting regularly with a group of students. The small group size allows for sessions to be tailored to the individual needs of the students and for each student to be involved in discussions.\footnote{Groups of only one student are usually avoided as this could be intimidating for the student. Furthermore, having an additional student provides some respite to think while the other is being questioned. The lone student would also miss out on the opportunity to hear the opinions of a peer.} Supervisions are generally structured around the problems, with the students given opportunity to ask about other areas of the course they were unsure of (or interested in); any spare time is used to discuss additional topics (perhaps the research of the supervisor). The supervision system is designed to encourage students to stay on top of their work and to provide them for the necessary support when this is difficult \citep[case study 12.1]{Gibbs2015}; its effectiveness at achieving this is manifest from the international reputation of Cambridge.

Tutorials in Birmingham are of a similar nature to supervisions. They differ in group size, which does not afford as intensive attention to individual students, and also in the amount of contact time. In second year, Cambridge's Natural Sciences undergraduates would have one hour per week per course (three hours per week in total), while Birmingham's Physics undergraduates have a single hour per week for all their courses.\footnote{The difference in teaching time is slightly ameliorated by the marginally longer teaching year in Birmingham: 22 weeks compared to 19.} A further difference is that tutorials in Birmingham include the (formal) teaching of key communication skills, such as preparing a curriculum vit\ae, giving a presentation and writing a scientific essay. These enhance the employability of the students (especially since communication skills are not traditionally emphasised within a physics course), but also absorb time that could have been spent on the main curriculum. While the tutorial system might not match supervisions in Cambridge, they still provide a great opportunity for students to engage with and ask questions of a member of research staff, an opportunity that is not present as part of all physics courses.

For any small-group teaching it is necessary for students to prepare. I did this by collaborating with my peers.

\section{Peer learning}\label{sec:peer}

As an undergraduate I would often work on problems with a group of fellow physicists. Doing so provided support for difficult questions, an opportunity to discuss concepts, and multiple points-of-view. It made the experience much more enjoyable, and motivated continued effort: working on questions was a social activity, and there was friendly competition to finding (the best) solutions. Cooperative learning has been found to improve student attainment \citep{Qin1995,Cabrera2002}, and this particular example can be considered as an informal version of peer instruction, which has been demonstrated to be effective in encouraging learning in physics \citep{Crouch2001,Pilzer2001,Miller2006}.

Not only did working as part of a group improve our understanding of physics, but it also enhanced our communication skills: we became accustomed to explaining physics and our solutions. This is a vital skill I have used in my own teaching.

\section{Translation from learning to teaching}

Several elements from my own studies have been incorporated into my current teaching style. From Cambridge I have inherited high expectations for my students' work. This helps stretch them to achieve their best and engage them by challenging them academically \citep{Bamber2015}. Cambridge also gave me much experience of discussing physics, both in supervisions and working outside. In my small-group teaching, engaging students in conversation is important: they need to actively discuss concepts, and ask questions freely. To aid this, I adopt a similar style to that I used when working with my peers. I employ humour, incorporating jokes and anecdotes. Use of humour can improve retention and ease anxiety \citep[e.g.,][and references therein]{Korobkin1988,Lesser2008}. The latter improves group rapport, which makes students more comfortable contributing to discussions and, in particular, asking questions on areas of weakness. While I cannot arrange for my students to benefit from the same collaborative learning that I did, I try to make sure that can obtain similar benefits from my teaching.

\bibliographystyle{physicsAuthorYearURL2}
\bibliography{teaching}

\end{document}
